\section{Motivation}
\label{section_intro_motivation}

Graph theory is a fundamental field of mathematics with significant relevance in
computer science due to its ability to model relationships between objects.
One of the key challenges in computer science is graph traversal and exploration, which involves systematically navigating through the nodes of a graph
with a specific purpose. This topic has practical applications across diverse fields, such as network routing, robotics, procedural generation, electronics design, and more.

When considering the complexities and time constraints imposed by real-world environments,
graph exploration must be extended to consider multi-agent systems. In such systems, multiple
autonomous agents collaborate to explore graphs, aiming to distribute tasks to achieve optimal efficiency. However, communication strategies
on these systems are challenging, as they need to balance the trade-off between the amount of information shared among agents
versus the quality of the solution. In other words, while communication can allow
agents to speed up their exploration, it may cost time and/or energy.

Various versions of this problem have been explored in research. 
One such approach, discussed by \citen{Kivelevitch2010}, proposed a generalization of Tarry's Algorithm with significant emphasis on minimizing data transfer.
However, the authors did not experiment with this solution on general graphs.

Despite the previously mentioned research, the concept of zero-communication exploration has not been
concretely established in the literature. We believe this gap is significant, as practical situations might involve
scenarios where communication is limited or impossible due to bandwidth limitations or energy consumption, such as deep-sea exploration,
search with energy-limited agents, or high-efficiency state space search.

\section{Objective}
\label{section_intro_objective}

This study aims to develop an efficient algorithm for graph exploration by multi-agents, specifically to find a goal node.
A previous paper by \citen{Arthur2023} discussed an approach for perfect maze exploration \cite{Naeem2021}, which can be modeled as a tree, a connected acyclic undirected graph.
Expanding the proposed algorithm to connected cyclic graphs enables the definition of a lower bound for exploration algorithms. This provides a benchmark for evaluating and comparing exploration strategies, allowing researchers to quantify the impact of communication on exploration efficiency relative to a baseline no-communication algorithm. This comparison is crucial for assessing the cost of implementation and the value brought by communication in contrast to a simpler, zero-communication strategy.

Alongside this main objective, this research seeks to incorporate heuristics and alternative strategies proven effective in zero-communication scenarios to enhance Tarry's algorithm. As outlined by \citen{Kivelevitch2010}, Tarry's algorithm relies on a random selection of neighbors for exploration, which may not always yield optimal results for different graph structures. Our goal was not only to analyze how these choices are made but also to investigate whether a secondary algorithm or heuristic could guide the selection process, thereby improving the overall efficiency of exploration.


\section{Definitions}
\label{section_intro_definitions}

This section aims to define key concepts in graph theory and
multi-agent exploration that are essential for understanding the context 
and methodology of our research, ensuring clarity and consistency in the subsequent discussions.

\subsection{Graph}
\label{section_definitions_graph}

Graphs are structures in mathematics and computer science used to represent
relationships between pairs of objects. According to \citen{Manber1989}, a graph
$G=(V,E)$ is defined as a set $V$ of vertices (or nodes) and a set $E$ of edges,
that connect pairs of vertices. 

\subsubsection{Common Nomenclature}

In graph theory, the basic terminology includes the following components:

\begin{itemize}
    \item Node (Vertex): The fundamental unit of a graph, representing an object or a point.
    \item Edge: A connection between a pair of vertices, representing a relationship between them.
    \item Adjacency: Two vertices $v$ and $w$ of a graph $G$ are adjacent if there is an edge $vw$ joining them, and the vertices $v$ and $w$ are then incident with such an edge. \cite{West2018}
    \item Degree: Degree of a vertex $v$ of $G$ is the number of edges incident with $v$. \cite{West2018}
    \item Path: A path in a graph $G$, as defined by \citen{West2018}, is a finite sequence of edges of the form $v_0v_1$, $v_1v_2$, ... in which any two consecutive edges are adjacent, all edges are distinct and all vertices $v_0$, $v_1$, ..., $v_m$ are distinct (except, possibly, $v_0 = v_m$).
    \item Cycle: A path where $v_0 = v_m$ and with at least one edge. \cite{West2018}
\end{itemize}

\subsubsection{Key Properties}

A few key properties of graphs that interest us are:

\begin{itemize}
    \item Connectivity: A graph is connected if and only if there is a path between each pair of vertices. \cite{West2018}
    \item Acyclicity: A graph is acyclic if it has no cycles. A tree is an acyclic connected graph.
\end{itemize}

Additionally, we differentiate between "perfect" and "imperfect" mazes:

\begin{itemize}
    \item Perfect Maze: A maze that contains no loops, meaning it can be represented as a tree (an acyclic connected graph). 
    \item Imperfect Maze: A maze that includes loops, making it representable as a general graph but not as a tree. 
\end{itemize}

\subsubsection{Traversal}

Graph traversal is the process of visiting all the vertices in a graph in a systematic manner. Two fundamental algorithms for graph traversal are:

\begin{itemize}
    \item Breadth-first search (BFS): Expands the frontier between discovered and undiscovered vertices uniformly across the breadth of the frontier. \cite{Manber1989}
    \item Depth-first search (DFS): Explores edges out of the most recently discovered vertex $v$ that still has unexplored edges leaving it. \cite{Manber1989}
    \item Tarry's Algorithm: Tarry's Algorithm is used to traverse the entire maze, or graph, in a depth first fashion, where each edge is travelled twice, and the agent finishes the motion at the same initial vertex it started from.\cite{Kivelevitch2010} This algorithm is described in the context of a physical maze, where the agent must physically move between adjacent nodes.
    It was first described over botanical mazes built as amusements for the XIX century aristocracy.
\end{itemize}

\subsection{Agent}
\label{section_definitions_agent}

An agent in maze exploration, as defined by \citen{Kivelevitch2010}, 
is an entity able to move within the maze in any direction as long as it does not encounter obstacles.

In the context of graph exploration, this concept extends to an agent's ability to traverse
edges between vertices within the graph structure. Each agent acts independently,
making decisions based on its individual memory and the local information available at its current vertex and its adjacent edges.


\subsection{Mixed Radix}
\label{section_definitions_mixed_radix}

A mixed radix system is a positional numerical system in which the numerical base for each digit
varies. This allows for more flexible representation of numbers,
accommodating various scales within the same numerical framework.

One mathematical approach for representing mixed radix numbers can be found in \citen{Arndt2011},
which establishes a comprehensive arithmetical method for manipulating them.  However,
for the specific subset of real numbers within the interval [0, 1], a more specialized approach is utilized, as described by \citen{Arthur2023}.

In this context, the mixed radix representation $A=[a_{0},a_{1},a_{2},...,a_{n-1}]$ 
of a number $x$ with respect to a radix vector $M=[m_{0},m_{1},m_{2},...,m_{n-1}]$,
where $x \in [0,1]$, is given by:

\begin{equation}
	x = \sum_{k=0}^{n-1} a_{k} \prod_{j=0}^{k} \frac{1}{m_j}
\end{equation}
where $a_j$ are non-negative integers, $m_j$ are integers such that $m_j \geq 2$, and $0 \leq a_j \leq m_j$. \cite{Arthur2023}


For instance, if $x$ is represented by $0.2_{3}1_{2}1_{4}0_{3}1_{5}$, and considering that $a_{i}$ is on the left of $a_{i+1}$, $x$ might be transformed to the decimal base by the following steps:

\begin{align}
& x=2_{3}1_{2}1_{4}0_{3}1_{5}\\
& A=[2,1,1,0,1]\\
& M=[3,2,4,3,5]\\
& n=5\\
& x = \sum_{k=0}^{4} a_{k} \prod_{j=0}^{k} \frac{1}{m_j} \approx 0.8777778
\end{align}

\subsubsection{Extending Mixed Radix with Unary Digits}
\label{section_definitions_unary_mixed_radix}

In our study, we extended the mixed radix system to include unary digits,
which are used to represent steps in a sequence where no meaningful decisions are made.
These unary digits do not contribute to the numerical value and can be omitted in the conversion to standard numerical forms.

Within our framework, the mixed radix representation $A=[a_{0},a_{1},a_{2},...,a_{n-1}]$ 
of a number $x$ with respect to a radix vector $M=[m_{0},m_{1},m_{2},...,m_{n-1}]$,
where $x \in [0,1]$, is given by:

Let  $U = \{ k \ |\ m_k \neq 1 \} $ be the set of indices where the base \( m_k \) is not unary.

\begin{equation}
    \label{eq:unary_mixed_radix}
	x = \sum_{k \in U} u_{k} \prod_{j=0}^{k} \frac{1}{m_j}
\end{equation}
where $a_k$ are non-negative integers, $m_j$ are integers such that $m_j \geq 1$, and $0 \leq a_j \leq m_j$.

Consider the mixed radix representation $x=0.2_{3}I_11_{2}I_1I_11_{4}0_{3}1_{5}$, where $I_1$ represents the unary digit and each $a_{i}$ is on the left of $a_{i+1}$. We can transform
$x$ to the decimal base using the following steps:

\begin{align}
    & x=2_{3}I_11_{2}I_1I_11_{4}0_{3}1_{5}\\
    & A=[2,I,1,I,I,1,0,1]\\
    & M=[3,1,2,1,1,4,3,5]\\
    & U=[0,2,5,6,7]\\
    & x = \sum_{k \in U} a_{k} \prod_{j=0}^{k} \frac{1}{m_j} \approx 0.8777778
\end{align}

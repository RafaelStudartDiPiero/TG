Graph theory, a pivotal field in mathematics and computer science, offers robust frameworks for modeling relationships and traversing graph structures.
This research delves into graph exploration,
crucial for applications like network routing, robotics, and procedural generation.
As real-world applications often involve complex environments with time constraints,
this study focuses on multi-agent systems for graph exploration,
where multiple autonomous agents collaborate to optimize task distribution.

This study's objective is to propose an efficient multi-agent algorithm for graph exploration without communication,
a crucial need in scenarios where communication is impractical or impossible,
such as deep-sea exploration or energy-constrained environments.
Building on the method for perfect maze exploration detailed by \citen{Arthur2023},
this study extends the approach to general graphs, that may include cycles.
Several algorithmic variations, combining the proposed approach and Tarry's variation \cite{Kivelevitch2010}, were implemented in an effort to compare how different algorithms perform in various scenarios and to identify efficient use cases. Furthermore, the algorithms were reimplemented using a generic graph library instead of a specialized perfect maze library \cite{Naeem2021}.

By structuring the exploration into well-defined and extensible classes,
the research offers a versatile framework for broader applications in multi-agent systems. This framework was used to evaluate different datasets, containing graphs with varying properties, paving the way for further advancements in autonomous graph exploration.

% TODO:(RAFAEL): Change this intro 
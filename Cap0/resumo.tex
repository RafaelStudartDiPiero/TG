A teoria dos grafos,
um campo fundamental da matemática e da ciência da computação,
oferece estruturas robustas para modelar relacionamentos e percorrer estruturas de grafos.
Esta pesquisa se aprofunda na exploração de grafos,
cruciais para aplicações como roteamento de redes,
robótica e geração procedural.
Como as aplicações no mundo real frequentemente envolvem ambientes complexos com restrições de tempo,
este estudo foca em sistemas multiagentes para a exploração de grafos,
onde múltiplos agentes autônomos colaboram para otimizar a distribuição de tarefas.

O objetivo deste estudo é propor um algoritmo eficiente de exploração de grafos por múltiplos agentes sem comunicação,
uma necessidade crucial em cenários onde a comunicação é impraticável ou impossível,
como exploração em águas profundas ou ambientes com restrição de energia.
Com base no método para exploração de labirintos perfeitos detalhado por \citen{Arthur2023},
este estudo estende a abordagem para grafos mais gerais, que podem incluir ciclos.
Além disso, os algoritmos foram reimplementados utilizando uma biblioteca geral de grafos e não uma biblioteca 
específica para labirintos perfeitos \cite{Naeem2021}. Ao estruturar a implementação em classes bem definidas e extensíveis,
a pesquisa oferece uma estrutura versátil para aplicações mais amplas em sistemas multiagentes,
abrindo caminho para novos avanços na exploração autônoma de grafos.
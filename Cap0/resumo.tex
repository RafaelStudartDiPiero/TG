A teoria dos grafos, um campo fundamental da matemática e da ciência da computação, oferece estruturas robustas para modelar relacionamentos e percorrer estruturas de grafos. Esta pesquisa se aprofunda na exploração de grafos, crucial para aplicações como roteamento de redes, robótica e geração procedural. Como as aplicações do mundo real frequentemente envolvem ambientes complexos com restrições de tempo, como operações de busca e resgate com drones, este estudo foca em sistemas multiagente para a exploração de grafos, onde múltiplos agentes autônomos colaboram para otimizar a distribuição de tarefas.

Este trabalho apresenta uma estrutura modular estruturada em classes extensíveis, facilitando a adição de novos algoritmos e conjuntos de dados. Ao ir além das limitações da exploração de labirintos perfeitos \cite{Naeem2021}, empregamos o NetworkX \cite{Hagberg2008}—uma biblioteca de grafos geral—que permite a análise de estruturas de grafos mais diversas. Essa flexibilidade possibilitou comparações em diversos conjuntos de dados para revelar padrões no desempenho dos algoritmos sob diferentes características dos grafos.

O objetivo deste estudo é propor um algoritmo multiagente eficiente para exploração de grafos sem comunicação, uma necessidade crucial em cenários onde a comunicação é impraticável ou impossível. Baseando-se no método de exploração de labirintos perfeitos detalhado por \citen{Arthur2023}, este estudo estende a abordagem para grafos gerais, incluindo aqueles com ciclos, e avalia o desempenho do algoritmo em quatro conjuntos de dados de complexidade variada. Ao estabelecer um benchmark para a eficiência da exploração, esta pesquisa permite uma avaliação dos custos e benefícios da incorporação da comunicação em estratégias multiagente, proporcionando uma base comparativa para o desenvolvimento futuro de algoritmos de exploração.

Além disso, esta pesquisa incorporou heurísticas e estratégias alternativas para aprimorar o algoritmo de Tarry, que tradicionalmente depende da seleção aleatória de vizinhos. As adaptações resultantes demonstraram melhorias de desempenho em conjuntos de dados específicos, ilustrando como processos de seleção direcionados podem otimizar a eficiência da exploração com base nas propriedades dos conjuntos de dados.
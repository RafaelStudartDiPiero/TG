This work presented an algorithm for multi-agent graph exploration without communication, expanding on the solution proposed by \citen{Arthur2023}. Given the flexibility of our implementation, we analyzed three different datasets beyond the original one by Arthur, validating our approach with previously obtained results and enabling a broader comparison across various datasets. This allowed us to investigate how different graph properties impact the performance of no-communication algorithms, establishing a baseline for exploration algorithms and providing a foundation to assess the impact of communication when evaluating distinct metrics.

Alongside defining the impact of communication on performance, we implemented various heuristics and enhancements to Tarry's algorithm, which yielded significant performance improvements in only a few specific conditions. These findings suggest that, while additional information can help guide the selection process, its effectiveness may be limited to particular scenarios that warrant further investigation.

Given the modular design of our framework, it can easily expand to new algorithms and datasets. This flexibility enables future evaluations across a wider range of datasets, helping determine if the performance patterns hold in different contexts and providing insight into effective strategies for specific exploration tasks.

In future research, introducing new metrics and test cases could deepen our understanding of the results, providing stronger support for our conjectures. This would help pinpoint the key properties that influence algorithm performance, offering a clearer basis for selecting the most suitable exploration strategy for each scenario, beyond relying solely on empirical analysis.

In addition to implementing new algorithms, future work could explore various communication models among agents. For instance, agents might only communicate when they occupy the same cell, which could provide a more cost-efficient strategy compared to maintaining a shared map. By measuring how different communication strategies impact performance, we can gain insights into their effectiveness in various applications, while also considering the costs associated with each communication method. 

Our current experiments did not address collisions between agents, allowing multiple agents to occupy the same node at the same time. This simplified assumption enabled a clearer focus on evaluating exploration strategies. However, future studies could address agent collisions to simulate more realistic scenarios and analyze how such interactions affect the exploration results.

Another area of interest lies in identifying particular cases where these algorithms achieve notably higher performance. By understanding the specific conditions under which each exploration strategy is most effective, we can target their practical applications in relevant fields more precisely. One straightforward application of zero-communication distributed search is the deployment of a swarm of drones over a topological map, allowing them to operate without expending energy on communication. This approach is viable even if the map is initially unknown, as long as it can be constructed consistently by all drones independently.

In conclusion, this research advances multi-agent graph exploration without communication, establishing benchmarks to evaluate the impact and performance of exploration algorithms and developing heuristics that significantly enhance established methods.
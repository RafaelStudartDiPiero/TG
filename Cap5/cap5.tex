In this study, the structures of all fifteen simulated materials were meticulously constructed. Each structure underwent rigorous validation and simulation using the VESTA software to ensure that every compound possessed the correct unit cell structure. Subsequently, the structural parameters were determined through energy minimization and relaxation, yielding results that exhibited excellent agreement with the experimental data, with the maximum deviation being $2.25\%$ from the reference values. Moreover, the band gaps, and electronic properties were computed using the Minushalf software, and the results were meticulously presented. In summary, the findings demonstrated a high level of concordance with the reference results, with the Root Mean Squared Error (RMSE) scores being three to two times lower than those obtained with DFT.

Therefore, it can be concluded that the software reproduces the results of the DFT -1/2 method, as demonstrated by the RMSE results presented in the previous section. Additionally, a comprehensive analysis was conducted to explain how the characteristics of the compounds may influence the results and identify the factors that can make reproducibility challenging to achieve. Thus, one can now combine the power of the DFT -1/2 method with the simplicity of the interface provided by Minushalf. This would popularize the method and enable the scientific community to benefit from better results for their analyses, potentially revolutionizing areas such as topological insulators, two-dimensional materials, and the study of solar panels.

Currently, GMSN is the main user of Minushalf and predominantly employs the software. It is expected that as a consequence of this developed work, the ``execute'' command will be increasingly used to obtain results and perform calculations that can be easily replicated without scripts that are often only known to the scientist. Regarding expansion to international audiences, the plan is to promote the software once we have obtained internal validation confirming its effectiveness, and even publish articles about its results. While this promotion may happen in the future, it is important to note that the software already has over 86 thousand downloads on PyPI and is being used in various locations around the globe.

Finally, there is also an initiative to expand Minushalf to Quantum Espresso with the aim of making the software compatible with open-source technologies used in first-principles calculations. It is believed that this expansion will further facilitate the adoption of Minushalf within the scientific community.